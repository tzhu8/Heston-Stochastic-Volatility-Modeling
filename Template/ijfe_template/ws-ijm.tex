%%%%%%%%%%%%%%%%%%%%%%%%%%%%%%%%%%%%%%%%%%%%%%%%%%%%%%%%%%%%%%%%%%%%%%%%%%%%
%% Trim Size: 9.75in x 6.5in
%% Text Area: 8in (include Runningheads) x 5in
%% ws-ijm.tex   :   8-11-05
%% Tex file to use with ws-ijm.cls written in Latex2E.
%% The content, structure, format and layout of this style file is the
%% property of World Scientific Publishing Co. Pte. Ltd.
%% Copyright 1995, 2002 by World Scientific Publishing Co.
%% All rights are reserved.
%%%%%%%%%%%%%%%%%%%%%%%%%%%%%%%%%%%%%%%%%%%%%%%%%%%%%%%%%%%%%%%%%%%%%%%%%%%%
%

\documentclass{ws-ijm}

\usepackage{graphicx}

\begin{document}

\markboth{Authors' Names}
{Instructions for Typesetting Camera-Ready Manuscripts}

%%%%%%%%%%%%%%%%%%%%% Publisher's Area please ignore %%%%%%%%%%%%%%
\catchline{}{}{}{}{}
%%%%%%%%%%%%%%%%%%%%%%%%%%%%%%%%%%%%%%%%%%%%%%%%%%%%%%%%%%%%%%%%%%%

\title{INSTRUCTIONS FOR TYPESETTING CAMERA-READY\\
MANUSCRIPTS USING COMPUTER SOFTWARE\footnote{Typeset title in
10~pt Times Roman uppercase and boldface. Please write
down in pencil a short title to be used as the running head.}}

\author{FIRST A. AUTHOR, SECOND B. AUTHOR and THIRD
C. AUTHOR\footnote{Typeset names in 8~pt Times Roman, uppercase
and lightface.  Use footnotes only to indicate if permanent and
present addresses are different. Funding information should go
in the Acknowledgement section.}}

\address{Full affiliations\footnote{Typeset
affiliation and mailing addresses in 8pt Times italic.} \\
,mailing addresses and telephone number}

\author{OTHER D. AUTHOR}

\address{Full affiliations \\
,mailing addresses and telephone number}

\maketitle

\begin{abstract}
Include a one-paragraph abstract of no more than 100 words. Do not include references, footnotes, or abbreviations in the abstract. Typeset the
abstract in 8 pt Times Roman with baselineskip of 10 pt, making
an indentation of $\frac14$ inch on the left and right margins.
Typeset similarly for keywords below.
\end{abstract}

\keywords{ Enclose with each manuscript, on a separate page, from three to five keywords. }


\section{General Appearance}	%) A SECTION HEADING
Contributions to the {\it International Journal of Mathematics}
will be reproduced by photographing the author's
submitted typeset manuscript.  It is therefore essential that
the manuscript be in its final form, and is an original computer
printout because it will be printed directly without any
editing. The manuscript should also be clean and unfolded. The
copy should be evenly printed on a high resolution printer (300
dots/inch or higher). If typographical errors cannot be avoided,
use cut and paste methods to correct them. Smudged copy, pencil
or ink text corrections will not be accepted. Do not use
cellophane or transparent tape on the surface as this interferes
with the picture taken by the publisher's camera.

\section{Style Guidelines for IJFE}


Author Information: Include the following information on the first page of the manuscript: (1) title, (2) author(s), (3) institutional affiliation, (4) address, and (5) telephone number.

Abstract: Include a one-paragraph abstract of no more than 100 words. Do not include references, footnotes, or abbreviations in the abstract.

Keywords: Enclose with each manuscript, on a separate page, from three to five keywords.

Typing Format: Double space with a minimum of 11pt fonts. Margins of at least 1in.

Headings and Subheadings: Use no more than three levels of headings. Begin all headings at the left margin and capitalize the first letter of the first word only. Headings should be numbered as, e.g., 1, 1.3, 2.4.5, etc.

Footnotes: Each footnote should appear at the bottom of the page on which it is cited in the text and should be indicated consecutively with superscript Arabic numerals.

Equations: Number consecutively only those equations that are referenced in the text. Indent equations and place numbers in parentheses at the right margin.

References: List references alphabetically by author's last name at the end of the text of the paper. They can be cited in the text as, e.g., "According to Smith and Jones (1995), ..." or "... (Smith and Jones, 1995)".

Tables: Type tables on separate pages after the references. Center the word "Table" followed by an Arabic numeral above the body of the table. Separate headings in a table from the title of the table and from the body of the table with solid lines. Verify that the text contains a reference to each table. When referring to a specific table in the text of the paper, use Table 1, Table 2, etc.

Figures: Figures must appear after the tables. Verify that the text contains a reference to each figure. When referring to specific figures in the text, use Fig. 1, Fig. 2, etc. When labeling figures, capitalize the first letter in the word and number with Arabic numerals (e.g., Figure 1). In figure titles, capitalize the first letter of the first word only. When supplying color figures, ensure that there is sufficient contrast to enable clear black and white printing. No figures will be printed in color.

\section{Major headings}
Headings and Subheadings: Use no more than three levels of headings. Begin all headings at the left margin and capitalize the first letter of the first word only. Headings should be numbered as, e.g., 1, 1.3, 2.4.5, etc.

\subsection{Sub-headings}
Sub-headings should be typeset in boldface italics.

\subsubsection{Sub-subheadings}
Sub-subheadings should be typeset in italics.

\subsection{Numbering and spacing}
Sections, sub-sections and sub-subsections are to be numbered in
Arabic.

\subsection{Lists of items}
Lists may be laid out with each item marked by a dot:

\begin{itemlist}
 \item item one,
 \item item two.
\end{itemlist}

%\begin{romanlist}[(ii)]
%\item item one
%\item item two
%	\begin{romanlist}[(b)]
%	\item Lists within lists can be numbered with lowercase
%              Roman letters,
%	\item second item.
%	\end{romanlist}
%\end{romanlist}

\section{Equations}
Displayed equations should be numbered consecutively in each
section, with the number set flush right and enclosed in
parentheses,

\noindent
\begin{equation}
\mu(n, t) = \frac{\sum^\infty_{i=1} 1(d_i < t,
N(d_i) = n)}{\int^t_{\sigma=0} 1(N(\sigma) = n)d\sigma}\,.
\label{this}
\end{equation}

Equations should be referred to in abbreviated form,
e.g."Eq.(\ref{this})" or "(4.1)". In multiple-line
equations, the number should be given on the last line.

Displayed equations are to be centered on the page width.
Standard English letters like x are to appear as $x$
(italicized) in the text if they are used as mathematical
symbols. Punctuation marks are used at the end of equations as
if they appeared directly in the text.

\begin{theorem}
Theorems$,$ lemmas$,$ etc. are to be numbered consecutively in the paper.
\end{theorem}

\begin{proof}
Proofs should end with a square.
\end{proof}

\section{Illustrations and Photographs}
Figures are to be inserted in the text nearest their first
reference. Original India ink drawings of glossy prints are
preferred. Please send one set of originals with copies. If the
author requires the publisher to reduce the figures, ensure that
the figures (including letterings and numbers) are large enough
to be clearly seen after reduction. If photographs are to be
used, only black and white ones are\break
acceptable.

\begin{figure}[th]
\centerline{\includegraphics[width=2.2in]{ijmf1}}
\vspace*{8pt}
\caption{Labeled tree {\it T}.\label{fig1}}
\end{figure}

Figures are to be sequentially numbered in Arabic numerals. The
caption must be placed below the figure. Typeset in 8~pt Times
Roman with baselineskip of 10~pt. Use double spacing between a
caption and the text that follows immediately.

Previously published material must be accompanied by written
permission from the author and publisher.

Figures should be referred to in the abbreviated form,
e.g.~``$\ldots$ in Fig.~\ref{fig1}'' or ``$\ldots$ in
Figs.~\ref{fig1} and 2''. Where the word ``Figure'' begins a
sentence, it should be spelt\break
in full.

\section{Tables}
Tables should be inserted in the text as close to the point of
reference as possible. Some space should be left above and below
the table.

\begin{table}[ht]
\tbl{Comparison of acoustic for frequencies for piston-cylinder problem.}
{\begin{tabular}{@{}cccc@{}} \toprule
Piston mass & Analytical frequency & TRIA6-$S_1$ model &
\% Error \\
& (Rad/s) & (Rad/s) \\ \colrule
1.0\hphantom{00} & \hphantom{0}281.0 & \hphantom{0}280.81 & 0.07 \\
0.1\hphantom{00} & \hphantom{0}876.0 & \hphantom{0}875.74 & 0.03 \\
0.01\hphantom{0} & 2441.0 & 2441.0\hphantom{0} & 0.0\hphantom{0} \\
0.001 & 4130.0 & 4129.3\hphantom{0} & 0.16\\ \botrule
\end{tabular}}
\end{table}

Tables should be numbered sequentially in the text in Arabic
numerals. Captions are to be centralized above the tables.
Typeset tables and captions in 8~pt Times Roman with
baselineskip of 10~pt.

If tables need to extend over to a second page, the continuation
of the table should be preceded by a caption,
e.g.~``Table~2. Cont'd.''

\section{References}
The format for references should be strictly followed.
References in the text are to be numbered in Arabic numerals.
They are to be cited in square brackets, \cite{2} before punctuation
marks. Standard journal abbreviations are preferred.

\section{Footnotes}
Footnotes should be numbered sequentially in superscript Arabic
numerals.\footnote{Footnotes should be typeset in 8~pt Times Roman
at the bottom of the page.}

\appendix

\section{Appendices}

Appendices should be used only when absolutely necessary. They
should come before Acknowledgments. If there is more than one
appendix, number them alphabetically.  Number displayed equations occurring in the Appendix
in this way, e.g.~(\ref{appeqn}), (A.2), etc.
\begin{equation}
f(j\delta, i\delta) \cong \frac{\pi}{M} \sum^M_{n-1}
Q_{\theta_n} (j\cos \theta_n + i\sin \theta_n)\, .\label{appeqn}
\end{equation}

\section*{Acknowledgments}
This section should come before the References. Funding
information may also be included here.

\begin{thebibliography}{0}


\bibitem{}Imai, J and KS Tan (2006). A general dimension reduction technique for derivative pricing, {\it Journal of Computational Finance}, 10(2), 129-155.

\bibitem{} Joe, S and FY Kuo (2008). Constructing Sobol' sequences with better two-dimensional projection, {\it SIAM Journal on Scientific Computing}, 30(5), 2635-2654.

\bibitem{} L'Ecuyer, P and C Lemieux (2002). Recent advances in randomized quasi-Monte Carlo methods, in M Dror, P L'Ecuyer and F Szidarovszki (editors),{\it  Modeling Uncertainty: An Examination of Stochastic Theory, Methods, and Applications}, pp. 419-474, Kluwer Academic, Boston.

\end{thebibliography}

\end{document}

