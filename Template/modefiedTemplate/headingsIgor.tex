%%% standard pachages
\usepackage{ amsmath, amssymb, amsfonts, graphicx, epsfig, mathtools, epstopdf}
%%%
\usepackage[section]{placeins}
\usepackage{morefloats}
\usepackage{float}
%%%%%%%%%%%%%%%%%%%%%%%%%%%%%%%%%%%%
%%% pachage to write Indicator function as 'blackboard bold 1': \mathbbm{1}. Made a custom macro \1=\mathbbm{1}
\usepackage{bbm}
%%%%%%%%%%%%%%%%%%%%%%%%%%%%%%%%%%%%

%%%%%%%%%%%%%%%%%%%%%%%%%%%%%%%%%%%%
% This package gives the enumerate environment an optional argument
% which determines the style in which the counter is printed.
% http://www.ctex.org/documents/packages/table/enumerate.pdf
\usepackage{enumerate}
%%%%%%%%%%%%%%%%%%%%%%%%%%%%%%%%%%%%

% to use strike out \sout
\usepackage[normalem]{ulem}


%%%%%%%%%%%%%%%%%%%%%%%%%%%%%%%%%%%%
% To display the labels used in a tex file in the dvi file (for example, if a theorem is labelled with the \label command) use the package
%\usepackage{showkeys}
%This will display all labels (for example, in the case of a labelled theorem, the label of the theorem will occur in the margin of the dvi file.
%The command above by itself not only displays labels when they are first named but also when they are cited (or referenced). To disable this feature use the following %options with the showkeys package
%\usepackage[notref, notcite]{showkeys}
%%%%%%%%%%%%%%%%%%%%%%%%%%%%%%%%%%%%


%%%%%%%%%%%%%%%%%%%%%%%%%%%%%%%%%%%%
% It extends the functionality of all the LATEX cross-referencing commands (including the table of contents, bibliographies etc) to produce \special commands which a driver can turn into hypertext links; it also provides new commands to allow the user to write ad hoc hypertext links, including those to external documents and URLs.
% http://www.tug.org/applications/hyperref/manual.html
\usepackage[colorlinks=true, pdfstartview=FitV, linkcolor=blue,
            citecolor=blue, urlcolor=blue]{hyperref}
\usepackage[usenames]{color}
\definecolor{Red}{rgb}{0.7,0,0.1}
\definecolor{Green}{rgb}{0,0.7,0}
%%%%%%%%%%%%%%%%%%%%%%%%%%%%%%%%%%%%


%%%%%%%%%%%%%%%%%%%%%%%%%%%%%%%%%%%%
% It is a LaTeX package to act as generalized
% interface for standard and non-standard bibliographic style files (BibTeX).
% http://www.ctan.org/tex-archive/macros/latex/contrib/natbib/
%\usepackage[numbers]{natbib}
%\usepackage{natbib}
%%%%%%%%%%%%%%%%%%%%%%%%%%%%%%%%%%%%


%%%%%%%%%%%%%%%%%%%%%%%%%%%%%%%%%%%%
% a good looking way to format urls
% http://mirror.its.uidaho.edu/pub/tex-archive/help/Catalogue/entries/url.html
\usepackage{url}
% Define a new 'leo' style for the package that will use a smaller font.
\makeatletter\def\url@leostyle{%
 \@ifundefined{selectfont}{\def\UrlFont{\sf}}{\def\UrlFont{\scriptsize\ttfamily}}} \makeatother\urlstyle{leo}
%%%%%%%%%%%%%%%%%%%%%%%%%%%%%%%%%%%%%

% to add accents and comments
%\usepackage{ comment}


%% custom margins
%\def\baselinestretch{1.1}
\setlength{\voffset}{-0.5in}
\setlength{\hoffset}{-0.5in}
\setlength{\textheight}{8.5in}
\setlength{\textwidth}{6in}
%% or use geometry pacakge
%\usepackage[margin=1.1in, dvips, letterpaper]{geometry}


%%%%%%%%%%%%%%%%%%%%%%%%%%%%%%%%%%%%%
%\newtheorem{theorem}{Theorem}[section]
%\newtheorem{proposition}[theorem]{Proposition}
%\newtheorem{lemma}{Lemma}[section]
%\newtheorem{corollary}[theorem]{Corollary}

%\theoremstyle{definition}
%\newtheorem{definition}[theorem]{Definition}
%\newtheorem{definition}{Definition}[section]
%\newtheorem{example}[theorem]{Example}
%\newtheorem{example}{Example}[section]
%\theoremstyle{remark}
%\newtheorem{remark}[theorem]{Remark}

%\newtheoremstyle{dotless}{}{}{\itshape}{}{\bfseries}{}{ }{}
%\theoremstyle{dotless}
%\newtheorem{assumption}{Assumption}
%\renewcommand*{\theassumption}{(\Alph{assumption})}
%\numberwithin{equation}{section}
%\numberwithin{theorem}{section}
%\renewcommand{\labelitemi}{ {\small $\rhd$}}

%%%%%%%%%%%%%%%%%%%%%%%%%%%%%%%%%%%%%
%%% used for editing and making comments in color
% Example \ig{Remarks and Commets}
\newcommand{\ti}[1]{\textcolor[rgb]{0.00, 0.0, 0.98}{T\&I: #1} }
\newcommand{\ig}[1]{\textcolor[rgb]{1.00, 0.0, 0.98}{Ig: #1} }
\newcommand{\ii}[1]{\textcolor[rgb]{0.00, 0.60, 0.5}{II: #1} }
\newcommand{\rr}[1]{\textcolor[rgb]{0.00, 0.65, 0}{RR: #1} }
\newcommand{\trb}[1]{\textcolor[rgb]{1.00, 0.00, 0}{TRB: #1} }
%%%%%%%%%%%%%%%%%%%%%%%%%%%%%%%%%%%%%


%%%%%%%%%%%%%%%%%%%%%%%%%%%%%%%%%%%%%
%%%     Igor's macros
%% \mathcal Letters
\def\cA{\mathcal{A}}
\def\cB{\mathcal{B}}
\def\cC{\mathcal{C}}
\def\cD{\mathcal{D}}
\def\cE{\mathcal{E}}
\def\cF{\mathcal{F}}
\def\cG{\mathcal{G}}
\def\cH{\mathcal{H}}
\def\cI{\mathcal{I}}
\def\cJ{\mathcal{J}}
\def\cK{\mathcal{K}}
\def\cL{\mathcal{L}}
\def\cM{\mathcal{M}}
\def\cN{\mathcal{N}}
\def\cO{\mathcal{O}}
\def\cP{\mathcal{P}}
\def\cQ{\mathcal{Q}}
\def\cR{\mathcal{R}}
\def\cS{\mathcal{S}}
\def\cT{\mathcal{T}}
\def\cU{\mathcal{U}}
\def\cV{\mathcal{V}}
\def\cW{\mathcal{W}}
\def\cX{\mathcal{X}}
\def\cY{\mathcal{Y}}
\def\cZ{\mathcal{Z}}

%% \mathbb Letters
\def\bA{\mathbb{A}}
\def\bB{\mathbb{B}}
\def\bC{\mathbb{C}}
\def\bD{\mathbb{D}}
\def\bE{\mathbb{E}}
\def\bF{\mathbb{F}}
\def\bG{\mathbb{G}}
\def\bH{\mathbb{H}}
\def\bI{\mathbb{I}}
\def\bJ{\mathbb{J}}
\def\bK{\mathbb{K}}
\def\bL{\mathbb{L}}
\def\bM{\mathbb{M}}
\def\bN{\mathbb{N}}
\def\bO{\mathbb{O}}
\def\bP{\mathbb{P}}
\def\bQ{\mathbb{Q}}
\def\bR{\mathbb{R}}
\def\bS{\mathbb{S}}
\def\bT{\mathbb{T}}
\def\bU{\mathbb{U}}
\def\bV{\mathbb{V}}
\def\bW{\mathbb{W}}
\def\bX{\mathbb{X}}
\def\bY{\mathbb{Y}}
\def\bZ{\mathbb{Z}}

\def\I{\mathbb{I}}
\def\1{\mathbbm{1}}


\newcommand{\pd}[1]{\partial_{#1}}      % partial derivative
\newcommand{\indFn}[1]{1 \! \! 1_{#1}}  % indicator function
\newcommand{\set}[1]{\left\{#1\right\}} % set: {xyz}
\renewcommand{\mid}{\;|\;}              % mid bar with small spaces before and after: x | y
\newcommand{\Mid}{\;\Big | \;}          % big bar with small spaces before and after:

\DeclareMathOperator*{\esssup}{ess\,sup} % ess sup
\DeclareMathOperator*{\essinf}{ess\,inf} % ess inf
\DeclareMathOperator*{\argmin}{arg\,min} % argmin
\DeclareMathOperator*{\argmax}{arg\,max} % argmax
%%%%%%%%%%%%%%%%%%%%%%%%%%%%%%%%%%%%%

%%%%%% d for derivative
\def\d{\mathrm{d}} 