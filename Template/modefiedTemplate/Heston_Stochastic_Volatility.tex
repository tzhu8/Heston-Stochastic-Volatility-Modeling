%%%%%%%%%%%%%%%%%%%%%%%%%%%%%%%%%%%%%%%%%%%%%%%%%%%%%%%%%%%%%%%%%%%%%%%%%%%%
%% Trim Size: 9.75in x 6.5in
%% Text Area: 8in (include Runningheads) x 5in
%% ws-ijfe.tex   :   8-11-05
%% Tex file to use with ws-ijm.cls written in Latex2E.
%% The content, structure, format and layout of this style file is the
%% property of World Scientific Publishing Co. Pte. Ltd.
%% Copyright 1995, 2002 by World Scientific Publishing Co.
%% All rights are reserved.
%%%%%%%%%%%%%%%%%%%%%%%%%%%%%%%%%%%%%%%%%%%%%%%%%%%%%%%%%%%%%%%%%%%%%%%%%%%%
%

\documentclass{ws-ijfe}

%%% standard pachages
\usepackage{ amsmath, amssymb, amsfonts, graphicx, epsfig, mathtools, epstopdf}
%%%
\usepackage[section]{placeins}
\usepackage{morefloats}
\usepackage{float}
%%%%%%%%%%%%%%%%%%%%%%%%%%%%%%%%%%%%
%%% pachage to write Indicator function as 'blackboard bold 1': \mathbbm{1}. Made a custom macro \1=\mathbbm{1}
\usepackage{bbm}
%%%%%%%%%%%%%%%%%%%%%%%%%%%%%%%%%%%%

%%%%%%%%%%%%%%%%%%%%%%%%%%%%%%%%%%%%
% This package gives the enumerate environment an optional argument
% which determines the style in which the counter is printed.
% http://www.ctex.org/documents/packages/table/enumerate.pdf
\usepackage{enumerate}
%%%%%%%%%%%%%%%%%%%%%%%%%%%%%%%%%%%%

% to use strike out \sout
\usepackage[normalem]{ulem}


%%%%%%%%%%%%%%%%%%%%%%%%%%%%%%%%%%%%
% To display the labels used in a tex file in the dvi file (for example, if a theorem is labelled with the \label command) use the package
%\usepackage{showkeys}
%This will display all labels (for example, in the case of a labelled theorem, the label of the theorem will occur in the margin of the dvi file.
%The command above by itself not only displays labels when they are first named but also when they are cited (or referenced). To disable this feature use the following %options with the showkeys package
%\usepackage[notref, notcite]{showkeys}
%%%%%%%%%%%%%%%%%%%%%%%%%%%%%%%%%%%%


%%%%%%%%%%%%%%%%%%%%%%%%%%%%%%%%%%%%
% It extends the functionality of all the LATEX cross-referencing commands (including the table of contents, bibliographies etc) to produce \special commands which a driver can turn into hypertext links; it also provides new commands to allow the user to write ad hoc hypertext links, including those to external documents and URLs.
% http://www.tug.org/applications/hyperref/manual.html
\usepackage[colorlinks=true, pdfstartview=FitV, linkcolor=blue,
            citecolor=blue, urlcolor=blue]{hyperref}
\usepackage[usenames]{color}
\definecolor{Red}{rgb}{0.7,0,0.1}
\definecolor{Green}{rgb}{0,0.7,0}
%%%%%%%%%%%%%%%%%%%%%%%%%%%%%%%%%%%%


%%%%%%%%%%%%%%%%%%%%%%%%%%%%%%%%%%%%
% It is a LaTeX package to act as generalized
% interface for standard and non-standard bibliographic style files (BibTeX).
% http://www.ctan.org/tex-archive/macros/latex/contrib/natbib/
%\usepackage[numbers]{natbib}
%\usepackage{natbib}
%%%%%%%%%%%%%%%%%%%%%%%%%%%%%%%%%%%%


%%%%%%%%%%%%%%%%%%%%%%%%%%%%%%%%%%%%
% a good looking way to format urls
% http://mirror.its.uidaho.edu/pub/tex-archive/help/Catalogue/entries/url.html
\usepackage{url}
% Define a new 'leo' style for the package that will use a smaller font.
\makeatletter\def\url@leostyle{%
 \@ifundefined{selectfont}{\def\UrlFont{\sf}}{\def\UrlFont{\scriptsize\ttfamily}}} \makeatother\urlstyle{leo}
%%%%%%%%%%%%%%%%%%%%%%%%%%%%%%%%%%%%%

% to add accents and comments
%\usepackage{ comment}


%% custom margins
%\def\baselinestretch{1.1}
\setlength{\voffset}{-0.5in}
\setlength{\hoffset}{-0.5in}
\setlength{\textheight}{8.5in}
\setlength{\textwidth}{6in}
%% or use geometry pacakge
%\usepackage[margin=1.1in, dvips, letterpaper]{geometry}


%%%%%%%%%%%%%%%%%%%%%%%%%%%%%%%%%%%%%
%\newtheorem{theorem}{Theorem}[section]
%\newtheorem{proposition}[theorem]{Proposition}
%\newtheorem{lemma}{Lemma}[section]
%\newtheorem{corollary}[theorem]{Corollary}

%\theoremstyle{definition}
%\newtheorem{definition}[theorem]{Definition}
%\newtheorem{definition}{Definition}[section]
%\newtheorem{example}[theorem]{Example}
%\newtheorem{example}{Example}[section]
%\theoremstyle{remark}
%\newtheorem{remark}[theorem]{Remark}

%\newtheoremstyle{dotless}{}{}{\itshape}{}{\bfseries}{}{ }{}
%\theoremstyle{dotless}
%\newtheorem{assumption}{Assumption}
%\renewcommand*{\theassumption}{(\Alph{assumption})}
%\numberwithin{equation}{section}
%\numberwithin{theorem}{section}
%\renewcommand{\labelitemi}{ {\small $\rhd$}}

%%%%%%%%%%%%%%%%%%%%%%%%%%%%%%%%%%%%%
%%% used for editing and making comments in color
% Example \ig{Remarks and Commets}
\newcommand{\ti}[1]{\textcolor[rgb]{0.00, 0.0, 0.98}{T\&I: #1} }
\newcommand{\ig}[1]{\textcolor[rgb]{1.00, 0.0, 0.98}{Ig: #1} }
\newcommand{\ii}[1]{\textcolor[rgb]{0.00, 0.60, 0.5}{II: #1} }
\newcommand{\rr}[1]{\textcolor[rgb]{0.00, 0.65, 0}{RR: #1} }
\newcommand{\trb}[1]{\textcolor[rgb]{1.00, 0.00, 0}{TRB: #1} }
%%%%%%%%%%%%%%%%%%%%%%%%%%%%%%%%%%%%%


%%%%%%%%%%%%%%%%%%%%%%%%%%%%%%%%%%%%%
%%%     Igor's macros
%% \mathcal Letters
\def\cA{\mathcal{A}}
\def\cB{\mathcal{B}}
\def\cC{\mathcal{C}}
\def\cD{\mathcal{D}}
\def\cE{\mathcal{E}}
\def\cF{\mathcal{F}}
\def\cG{\mathcal{G}}
\def\cH{\mathcal{H}}
\def\cI{\mathcal{I}}
\def\cJ{\mathcal{J}}
\def\cK{\mathcal{K}}
\def\cL{\mathcal{L}}
\def\cM{\mathcal{M}}
\def\cN{\mathcal{N}}
\def\cO{\mathcal{O}}
\def\cP{\mathcal{P}}
\def\cQ{\mathcal{Q}}
\def\cR{\mathcal{R}}
\def\cS{\mathcal{S}}
\def\cT{\mathcal{T}}
\def\cU{\mathcal{U}}
\def\cV{\mathcal{V}}
\def\cW{\mathcal{W}}
\def\cX{\mathcal{X}}
\def\cY{\mathcal{Y}}
\def\cZ{\mathcal{Z}}

%% \mathbb Letters
\def\bA{\mathbb{A}}
\def\bB{\mathbb{B}}
\def\bC{\mathbb{C}}
\def\bD{\mathbb{D}}
\def\bE{\mathbb{E}}
\def\bF{\mathbb{F}}
\def\bG{\mathbb{G}}
\def\bH{\mathbb{H}}
\def\bI{\mathbb{I}}
\def\bJ{\mathbb{J}}
\def\bK{\mathbb{K}}
\def\bL{\mathbb{L}}
\def\bM{\mathbb{M}}
\def\bN{\mathbb{N}}
\def\bO{\mathbb{O}}
\def\bP{\mathbb{P}}
\def\bQ{\mathbb{Q}}
\def\bR{\mathbb{R}}
\def\bS{\mathbb{S}}
\def\bT{\mathbb{T}}
\def\bU{\mathbb{U}}
\def\bV{\mathbb{V}}
\def\bW{\mathbb{W}}
\def\bX{\mathbb{X}}
\def\bY{\mathbb{Y}}
\def\bZ{\mathbb{Z}}

\def\I{\mathbb{I}}
\def\1{\mathbbm{1}}


\newcommand{\pd}[1]{\partial_{#1}}      % partial derivative
\newcommand{\indFn}[1]{1 \! \! 1_{#1}}  % indicator function
\newcommand{\set}[1]{\left\{#1\right\}} % set: {xyz}
\renewcommand{\mid}{\;|\;}              % mid bar with small spaces before and after: x | y
\newcommand{\Mid}{\;\Big | \;}          % big bar with small spaces before and after:

\DeclareMathOperator*{\esssup}{ess\,sup} % ess sup
\DeclareMathOperator*{\essinf}{ess\,inf} % ess inf
\DeclareMathOperator*{\argmin}{arg\,min} % argmin
\DeclareMathOperator*{\argmax}{arg\,max} % argmax
%%%%%%%%%%%%%%%%%%%%%%%%%%%%%%%%%%%%%

%%%%%% d for derivative
\def\d{\mathrm{d}} 

\usepackage{graphicx}


\begin{document}

\markboth{Authors' Names}
{Instructions for Typesetting Camera-Ready Manuscripts}

%%%%%%%%%%%%%%%%%%%%% Publisher's Area please ignore %%%%%%%%%%%%%%
\catchline{}{}{}{}{}
%%%%%%%%%%%%%%%%%%%%%%%%%%%%%%%%%%%%%%%%%%%%%%%%%%%%%%%%%%%%%%%%%%%

\title{Computing Option Prices Based on Heston Model to a Specified Tolerance\footnote{Typeset title in
10~pt Times Roman uppercase and boldface. Please write
down in pencil a short title to be used as the running head.}}

\author{FIRST A. AUTHOR, SECOND B. AUTHOR and THIRD
C. AUTHOR\footnote{Typeset names in 8~pt Times Roman, uppercase and lightface.  Use footnotes only to indicate if permanent and present addresses are different. Funding information should go in the Acknowledgement section.}}

\address{Full affiliations\footnote{Typeset
affiliation and mailing addresses in 8pt Times italic.} \\
,mailing addresses and telephone number}

\author{OTHER D. AUTHOR}

\address{Full affiliations \\
,mailing addresses and telephone number}

\maketitle

\begin{abstract}
Include a one-paragraph abstract of no more than 100 words. Do not include references, footnotes, or abbreviations in the abstract. Typeset the
abstract in 8 pt Times Roman with baselineskip of 10 pt, making
an indentation of $\frac14$ inch on the left and right margins.
Typeset similarly for keywords below.
\end{abstract}

\keywords{ Enclose with each manuscript, on a separate page, from three to five keywords. }

\section{Introduction}

The Guaranteed Automatic Integration Library (GAIL) is a suite of algorithms that includes Monte Carlo methods for multidimensional integration and computation of means, which is developed by Professor Hickernell and his collaborators. GAIL has theoretical guarantees that the user-defined error tolerance will be met. The financial application module of GAIL is under construction, and our work is aimed to add algorithms of calculating stochastic volatility model in the asset path class. There are several well-known stochastic volatility models: the Hull-white model (1987), the Scott-Chesny model (1989), the Heston model (1993) and the SABR model (2002). The Heston model is of particular interest to us since it is one of the most widely used stochastic volatility models.


We apply the Quadratic Exponential (QE) algorithm to simulate the volatility process and use the Broadie-Kaya scheme (2006) to discretize the asset price process. The QE model was developed by Andersen (2006). It is a market standard simulation method for the Heston model. Its attractiveness lies in its efficiency. It relies on simple probability density functions and needs a moderate amount of storage. Depending on the value of the volatility, the QE scheme approximates its distribution using either Gaussian or exponential distribution. After getting the value of the volatility for each time step, we discretize the asset price process. As we know, the computation of Broadie-Kaya algorithm is time consuming, but it is bias-free by construction. We discretize its scheme to get the simulation of the dynamic of asset price process. The Broadie-Kaya scheme does not satisfy an equivalent discrete-time martingale condition. The martingale property can be attained by adjusting a certain term in the scheme.


The contribution of this paper is twofold. First, the implementation of the QE scheme in GAIL so that we get the simulation algorithm of Heston model with guaranteed automatically realized user-defined error tolerance. When setting the volatility of the volatility of the asset price process as zero, the algorithm works like the algorithm of asset price process with deterministic volatility. For deterministic volatility case, we compare the results with other simulation methods such as geometric Brownian motion and quasi-Monte Carlo. It shows that the new algorithm is accurate and fast. Second, we apply variance reduction techniques such as importance sampling, control variates and Quasi-Monte Carlo methods to increase accuracy. The advantages of the new algorithm is strengthened.


The setup of the paper is as follows: we first introduce Heston model, QE scheme and the brief idea behind GAIL. Then, we explain our new algorithm and the improved algorithm using variance reduction techniques. At last, we show its performance in comparison with various widespread used schemes and with different variance reduction applications

\section{Options Modeled by the Heston Stochastic Volatility Model}
\subsection{Heston Model}
Heston model is defined as
 \begin{align}
    \d X & =\mu X\,\d t+\sqrt{V}X\,\d W_1\label{eq1}\\
    \d V &= -\kappa (\theta-V)\,\d t +\nu\sqrt{V}\,\d W_2\label{eq2}
  \end{align}
\begin{align*}
\text{where }&\\
   & \d W_1 \text{ and } \d W_2 \text{ are two standard Brownian motions which are correlated}\\
   &\rho \text{ is the correlation,} \\
   & \mu \text{ is the the risk-free interest rate,}\\%r \text{ and } d \text{ are respectively the continuously compounded risk-free interest rate and dividend yield,}  \\
   & \kappa \text{ is the speed of mean reversion,} \\
   & \theta \text{ is the value of the long-term variance,} \\
   & \nu \text{ is the volatility of volatility.}
\end{align*}

Applying the Ito's formula to \eqref{eq1}, an equivalent form to simulate the process of asset price is shown in Eq.(\ref{logX}).
\begin{equation}\label{logX}
  \d\ln(X)=(\mu-\frac{V}{2})\,\d t+\sqrt{V}\,\d W_1
\end{equation}

\subsection{Quadratic Exponential Scheme}
We applied the Quadratic Exponential Scheme illustrated in Andersen (2006) \cite{Andersen} to simulate the volatility process. 
\begin{enumerate}
\item Given $\hat{V}(t)$, compute $m$ and $s^2$ from following equations
\begin{align*}
  m &=\theta + (\hat{V}(t)-\theta)e^{-\kappa\Delta} \\
  s^2 &=\frac{\hat{V}(t)\nu^2 e^{-\kappa\Delta}}{\kappa}\bigg(1-e^{-\kappa\Delta}\bigg)+\frac{\theta\nu^2}{2\kappa}\bigg(1-e^{-\kappa\Delta}\bigg)^2
\end{align*}
\item Compute $\psi=s^2/m^2$\\
\item Draw a uniform random number $U_V$
\item If $\psi\leq\psi_c$:
\begin{enumerate}
\item Compute $a$ and $b$ from following equations
\begin{align*}
  b^2 & =2\psi^{-1}-1+\sqrt{2\psi^{-1}}\sqrt{2\psi^{-1}-1}\geq 0 \\
  a & =\frac{m}{1+b^2}
\end{align*}
\item Compute $Z_V=\Phi^{-1}(U_V)$
\item Set $\hat{V}(t+\Delta)=a(b+Z_V)^2$
\end{enumerate}
\item Otherwise, if $\psi>\psi_c$
\begin{enumerate}
  \item Compute $\beta$ and $p$ according to equations
  \begin{align*}
    p & =\frac{\psi-1}{\psi+1}\in[0,1) \\
    \beta & =\frac{1-p}{m}=\frac{2}{m(\psi+1)}>0
  \end{align*}
  \item set $\hat{V}(t+\Delta)=\Psi^{-1}(U_V;p,\beta)$
  \begin{equation*}
    \Psi(x) = \text{Pr}(\hat{V}(t+\Delta)\leq x) = p+(1-p)(1-e^{-\beta x}),\, x\geq 0
  \end{equation*}
  \[
  \Psi^{-1}(u)=\Psi^{-1}(u;p,\beta)=
  \begin{cases}
    0,\,&0\leq u \leq p,\\
    \beta^{-1}\ln(\frac{1-p}{1-u}),&p<u\leq1
  \end{cases}
  \]
\end{enumerate}
\end{enumerate}

\subsection{Broadie-Kaya discretization scheme for $\hat{X}$}
First we integrate the SDE for $V(t)$ to have a bias-free scheme,
\begin{equation*}
  V(t+\Delta) = V(t)+\int_{t}^{t+\Delta}\kappa(\theta-V(u))\,\d u +\nu\int_{t}^{t+\Delta}\sqrt{V(u)}\,\d W_V(u)
\end{equation*}
and it can be written as
\begin{equation}\label{dW_V}
\int_{t}^{t+\Delta}\sqrt{V(u)}\,\d W_V(u)=\nu^{-1}\bigg(V(t+\Delta)- V(t)-\int_{t}^{t+\Delta}\kappa(\theta-V(u))\,\d u\bigg)
\end{equation}
Recall \eqref{logX}, by Cholesky decomposition, we have
\begin{equation*}
 \d \ln X(t)=(\mu-\frac{1}{2}V(t))\,\d t +\sqrt{V(t)}\big(\rho\,\d W_V(t)+\sqrt{1-\rho^2}\,\d W(t)\big)
\end{equation*}
where $W$ is a Brownian motion independent of $W_V$.

Now we integrate the above equation,
\begin{equation*}
\begin{split}
  \ln X(t+\Delta)=&\ln X(t) +\mu\Delta-\frac{1}{2}\int_{t}^{t+\Delta}V(u)\,\d u +\rho\int_{t}^{t+\Delta}\sqrt{V(u)}\,\d W_V(u)\\ &+\sqrt{1-\rho^2}\int_{t}^{t+\Delta}\sqrt{V(u)}\,\d W(u)
\end{split}
\end{equation*}
Substituting Eq.\eqref{dW_V} into it, we get
\begin{equation}\label{lnX}
\begin{split}
   \ln X(t+\Delta)=&\ln X(t)+\mu\Delta+\frac{\rho}{\nu}(V(t+\Delta)-V(t)-\kappa\theta\Delta)+\bigg(\frac{\kappa\rho}{\nu}-\frac{1}{2}\bigg)\int_{t}^{t+\Delta}V(u)\,\d u\\
    &+\sqrt{1-\rho^2}\int_{t}^{t+\Delta}\sqrt{V(u)}\,\d W(u).
\end{split}
\end{equation}
We need to find appropriate approximation for the integrals in Eq.\eqref{lnX}. For now, we simply write
\begin{equation}\label{approx_du}
  \int_{t}^{t+\Delta}V(u)\,\d u \approx \Delta[\gamma_1 V(t)+\gamma_2 V(t+\Delta)]
\end{equation}
Conditioning on $V(t)$ and $\int_{t}^{t+\Delta}V(u)\,\d u$, the It$\hat{\text{o}}$ integral
\begin{equation*}
  \int_{t}^{t+\Delta}\sqrt{V(u)}\,\d W(u)
\end{equation*}
is Gaussian with mean zero and variance $\int_{t}^{t+\Delta}V(u)\,\d u$. So, we write
\begin{equation*}
  \int_{t}^{t+\Delta}\sqrt{V(u)}\,\d W(u)\approx \Delta\sqrt{\gamma_1V(t)+\gamma_2V(t+\Delta)}\cdot Z
\end{equation*}
Therefore, we have the following discretiztion scheme
\begin{equation}\label{eq3}
  \text{ln}\hat{X}(t+\Delta)=\text{ln}\hat{X}(t)+K_0+K_1\hat{V}(t)+K_2\hat{V}(t+\Delta)+\sqrt{K_3\hat{V}(t)+K_4\hat{V}(t+\Delta)}\cdot Z
\end{equation}
where $Z$ is a standard Gaussian random variable, independent of $\hat{V}$, and $K_0,\dots,K_4$ are given by
\begin{align*}
  K_0&=-\frac{\rho\kappa\theta}{\nu}\Delta &
  K_1&=\gamma_1\Delta\bigg(\frac{\kappa\rho}{\nu}-\frac{1}{2}\bigg)-\frac{\rho}{\nu},\\ K_2&=\gamma_2\Delta\bigg(\frac{\kappa\rho}{\nu}-\frac{1}{2}\bigg)+\frac{\rho}{\nu}, & K_3&=\gamma_1\Delta(1-\rho^2), \\  K_4&=\gamma_2\Delta(1-\rho^2).
\end{align*}
where $\gamma_1=\gamma_2=\frac{1}{2}$ in Anderson (2006).

\subsubsection{Implementation steps}

With given values of $\gamma_1$ and $\gamma_2$ and combined with the simulation scheme of $V$, the discretization scheme for ln$X$ can be generated in the following fashion:
\begin{enumerate}
  \item Given $\hat{V}(t)$, generate $\hat{V}(t+\Delta)$ using QE schemes
  \item Draw a uniform random number $U$, independent of all random numbers used for $\hat{V}(t+\Delta)$
  \item Set $Z=\Phi^{-1}(U)$
  \item Given ln$\hat{X}(t)$, $\hat{V}(t)$ and the value for $\hat{V}(t+\Delta)$ computed in Step 1, compute ln$\hat{X}(t+\Delta)$ from Eq.\eqref{eq3}
\end{enumerate}

\subsubsection{Martingale correction}
Under the QE scheme, a martingale correction scheme is illustrated in Andersen (2006).
%\begin{equation*}
%E(\hat{X}(t+\Delta)|\hat{X}(t))=\hat{X}e^{K^*_0+(K_1+\frac{1}{2}K_3)\hat{V}(t)}E(e^{(K_2+\frac{1}{2}K_4)\hat{V}(t+\Delta)})=\hat{X}
%\end{equation*}

\begin{equation*}
  \text{ln}\hat{X}(t+\Delta)=\text{ln}\hat{X}(t)+K_0^*+K_1\hat{V}(t)+K_2\hat{V}(t+\Delta)+\sqrt{K_3\hat{V}(t)+K_4\hat{V}(t+\Delta)}\cdot Z
\end{equation*}
where
\[
  K_0^*=
  \begin{cases}
    -\frac{Ab^2a}{1-2Aa}+\frac{1}{2}\text{ln}(1-2Aa)-(K_1+\frac{1}{2}{K_3}){\hat{V}(t)},\,&\psi\leq\psi_c,\\
    -\text{ln}\bigg(\frac{\beta(1-p)}{\beta-A}\bigg)-(K_1+\frac{1}{2}{K_3}){\hat{V}(t)},&\psi>\psi_c
  \end{cases}
  \]
and $A=K_2+\frac{1}{2}K_4$.
\section{Overcoming Numerical Errors for Samll Volatility of Volatility}

\section{Determining the Number of Samples Required to Meet a Specified Error Tolerance}

\section{Numerical Examples}

\section{Discussion}

\newpage

%\section{General Appearance}	%) A SECTION HEADING


%\section{Style Guidelines for IJFE}


%Author Information: Include the following information on the first page of the manuscript: (1) title, (2) author(s), (3) institutional affiliation, (4) address, and (5) telephone number.

%Abstract: Include a one-paragraph abstract of no more than 100 words. Do not include references, footnotes, or abbreviations in the abstract.

%Keywords: Enclose with each manuscript, on a separate page, from three to five keywords.

%Typing Format: Double space with a minimum of 11pt fonts. Margins of at least 1in.

%Headings and Subheadings: Use no more than three levels of headings. Begin all headings at the left margin and capitalize the first letter of the first word only. Headings should be numbered as, e.g., 1, 1.3, 2.4.5, etc.

%Footnotes: Each footnote should appear at the bottom of the page on which it is cited in the text and should be indicated consecutively with superscript Arabic numerals.

%Equations: Number consecutively only those equations that are referenced in the text. Indent equations and place numbers in parentheses at the right margin.

%References: List references alphabetically by author's last name at the end of the text of the paper. They can be cited in the text as, e.g., "According to Smith and Jones (1995), ..." or "... (Smith and Jones, 1995)".

%Tables: Type tables on separate pages after the references. Center the word "Table" followed by an Arabic numeral above the body of the table. Separate headings in a table from the title of the table and from the body of the table with solid lines. Verify that the text contains a reference to each table. When referring to a specific table in the text of the paper, use Table 1, Table 2, etc.

%Figures: Figures must appear after the tables. Verify that the text contains a reference to each figure. When referring to specific figures in the text, use Fig. 1, Fig. 2, etc. When labeling figures, capitalize the first letter in the word and number with Arabic numerals (e.g., Figure 1). In figure titles, capitalize the first letter of the first word only. When supplying color figures, ensure that there is sufficient contrast to enable clear black and white printing. No figures will be printed in color.

%\section{Major headings}
%Headings and Subheadings: Use no more than three levels of headings. Begin all headings at the left margin and capitalize the first letter of the first word only. Headings should be numbered as, e.g., 1, 1.3, 2.4.5, etc.

%\subsection{Sub-headings}
%Sub-headings should be typeset in boldface italics.

%\subsubsection{Sub-subheadings}
%Sub-subheadings should be typeset in italics.

%\subsection{Numbering and spacing}
%Sections, sub-sections and sub-subsections are to be numbered in
%Arabic.

%\subsection{Lists of items}
%Lists may be laid out with each item marked by a dot:

%\begin{itemlist}
% \item item one,
 %\item item two.
%\end{itemlist}

%\begin{romanlist}[(ii)]
%\item item one
%\item item two
%	\begin{romanlist}[(b)]
%	\item Lists within lists can be numbered with lowercase
%              Roman letters,
%	\item second item.
%	\end{romanlist}
%\end{romanlist}

%\section{Equations}
%Displayed equations should be numbered consecutively in each section, with the number set flush right and enclosed in parentheses,


%\begin{theorem}
%Theorems$,$ lemmas$,$ etc. are to be numbered consecutively in the paper.
%\end{theorem}

%\begin{proof}
%Proofs should end with a square.
%\end{proof}

%\section{Illustrations and Photographs}
%Figures are to be inserted in the text nearest their first reference. Original India ink drawings of glossy prints are preferred. Please send one set of originals with copies. If the author requires the publisher to reduce the figures, ensure that the figures (including letterings and numbers) are large enough to be clearly seen after reduction. If photographs are to be used, only black and white ones are\break acceptable.

%\begin{figure}[th]
%\centerline{\includegraphics[width=2.2in]{ijmf1}}
%\vspace*{8pt}
%\caption{Labeled tree {\it T}.\label{fig1}}
%\end{figure}

%Figures are to be sequentially numbered in Arabic numerals. The caption must be placed below the figure. Typeset in 8~pt Times Roman with baselineskip of 10~pt. Use double spacing between a caption and the text that follows immediately.

%Previously published material must be accompanied by written permission from the author and publisher.

%Figures should be referred to in the abbreviated form, e.g.~``$\ldots$ in Fig.~\ref{fig1}'' or ``$\ldots$ in Figs.~\ref{fig1} and 2''. Where the word ``Figure'' begins a sentence, it should be spelt\break in full.

%\section{Tables}
%Tables should be inserted in the text as close to the point of reference as possible. Some space should be left above and below the table.

%\begin{table}[ht]
%\tbl{Comparison of acoustic for frequencies for piston-cylinder problem.}
%{\begin{tabular}{@{}cccc@{}} \toprule
%Piston mass & Analytical frequency & TRIA6-$S_1$ model &
%\% Error \\
%& (Rad/s) & (Rad/s) \\ \colrule
%1.0\hphantom{00} & \hphantom{0}281.0 & \hphantom{0}280.81 & 0.07 \\
%0.1\hphantom{00} & \hphantom{0}876.0 & \hphantom{0}%875.74 & 0.03 \\
%0.01\hphantom{0} & 2441.0 & 2441.0\hphantom{0} & %0.0\hphantom{0} \\
%0.001 & 4130.0 & 4129.3\hphantom{0} & 0.16\\ \botrule
%\end{tabular}}
%\end{table}

%Tables should be numbered sequentially in the text in Arabic numerals. Captions are to be centralized above the tables. Typeset tables and captions in 8~pt Times Roman with baselineskip of 10~pt.

%If tables need to extend over to a second page, the continuation of the table should be preceded by a caption, e.g.~``Table~2. Cont'd.''

%\section{References}
%The format for references should be strictly followed. References in the text are to be numbered in Arabic numerals. They are to be cited in square brackets, \cite{2} before punctuation marks. Standard journal abbreviations are preferred.

%\section{Footnotes}
%Footnotes should be numbered sequentially in superscript Arabic numerals.\footnote{Footnotes should be typeset in 8~pt Times Roman at the bottom of the page.}

%\appendix

%\section{Appendices}

%Appendices should be used only when absolutely necessary. They should come before Acknowledgments. If there is more than one appendix, number them alphabetically.  Number displayed equations occurring in the Appendix in this way, e.g.~(\ref{appeqn}), (A.2), etc.
%\begin{equation}
%f(j\delta, i\delta) \cong \frac{\pi}{M} \sum^M_{n-1}
%Q_{\theta_n} (j\cos \theta_n + i\sin \theta_n)\, .\label{appeqn}
%\end{equation}

%\section*{Acknowledgments}
%This section should come before the References. Funding
%information may also be included here.

\begin{thebibliography}{99}
\bibitem{Andersen}Andersen, Leif B. G. (2006). Efficient Simulation of the Heston Stochastic Volatility Model.

%\bibitem{}Imai, J and KS Tan (2006). A general dimension reduction technique for derivative pricing, {\it Journal of Computational Finance}, 10(2), 129-155.

%\bibitem{} Joe, S and FY Kuo (2008). Constructing Sobol' sequences with better two-dimensional projection, {\it SIAM Journal on Scientific Computing}, 30(5), 2635-2654.

%\bibitem{} L'Ecuyer, P and C Lemieux (2002). Recent advances in randomized quasi-Monte Carlo methods, in M Dror, P L'Ecuyer and F Szidarovszki (editors),{\it  Modeling Uncertainty: An Examination of Stochastic Theory, Methods, and Applications}, pp. 419-474, Kluwer Academic, Boston.

\end{thebibliography}

\end{document}

